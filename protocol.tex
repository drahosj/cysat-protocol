%commands.tex - command protocol document for CySat

% Jake Drahos - Januar 2014

\documentclass{article}
\usepackage{times}
\usepackage[margin=0.75in]{geometry}

\begin{document}

\title{CySat Command and Data Protocol Description}
\author{Jake Drahos \\ \texttt{drahos@iastate.edu}}
\date{\today\\v0.2}


\newcommand{\mquery}{\texttt{QUERY }}
\newcommand{\mresult}{\texttt{RESULT }}
\newcommand{\mcommand}{\texttt{COMMAND }}
\newcommand{\mdownlink}{\texttt{DOWNLINK }}
\newcommand{\macommand}{\texttt{ACK\_COMMAND }}
\newcommand{\madownlink}{\texttt{ACK\_DOWNLINK }}
\newcommand{\merror}{\texttt{NACK\_ERROR }}


\maketitle

\section{Introduction}
This document is the canonical reference for the communications protocol used for CySat I and
future satellites for the CySat project.
This document is currently a working draft. As such, it is incomplete. Additions will be made, however, major
changes to already established details will not occur except in large version number upgrades of this document (for example,
version 0.7 to 1.1, but not from 0.4 to 0.5).
Version history can be found in the last section of this document.

\section{Protocol Overview}
The protocol is based on the idea of \emph{messages}.
Messages are entirely in ASCII.
Each message has several \emph{fields}, each of which is separated by a comma.
Every message begins with a start character, and ends with a stop character.

The communications protocol involves several types of message:
\begin{center}
  \begin{tabular}{| l | l | l |}
    \hline
    Message Type & Usage & Origin \\ \hline
    \mquery & Query the satellite & Base \\
    \mresult & Contains return information from a query & Satellite \\ \hline
    \mcommand & Issues a command to the satellite & Base \\
    \macommand & Acknowledges that a command was performed & Satellite \\ \hline
    \mdownlink & Downlink packet & Satellite \\
    \madownlink & Acknowledges downlink packet & Base \\ \hline
    \merror & Acknowledges an invalid packet & Satellite \\
    \hline
  \end{tabular}
\end{center}
  
  
\section{Message String Format}
  A valid command string takes the following format: \\[5pt]
  \verb=!MESSAGE_TYPE,FIELD_1,FIELD_2,CHECKSUM$=\\[5pt]
  There are three reserved characters, which cannot appear anywhere in an ASCII command:
  \begin{itemize}
  \item \verb=!= --- The start character
  \item \verb=$= --- The stop character
  \item \verb=,= --- Field separator
  \end{itemize}
  
  
  Each different message type has different requirements for valid lengths.
  
  
\begin{center}
  \begin{tabular}{| l | l | c |}
    \hline
    Message Type & Minimum Fields & Count \\ \hline
    \mquery & Type, Query Subtype, Checksum & 3 \\
    \mcommand & Type, Command Subtype, Checksum & 3 \\
    \mdownlink & Type, ID, Data, Checksum & 4 \\
    \mresult & Type, Query Subtype, Result, Checksum  & 4 \\
    \macommand & Type, Command Subtype, Checksum & 3 \\
    \madownlink & Type, ID, Checksum & 3 \\
    \merror & Type, Subtype, Checksum & 3 \\
    \hline
  \end{tabular}
\end{center}

Note that these are minimum required fields. Some messages, such as \mquery and \mcommand may have
different required parameters depending on the exact command or query being executed.

The first field of any message is known as the ``Type'' field, and corresponds to the message type.
The second field is often the message ``subtype''. For \mquery and \mcommand messages, this is the `command' or `query'
being performed. For \mresult and \macommand messages, this corresponds to the subtype of the message
being acknowledged or the query being answered. The last field is always a checksum.


  The checksum is a one-byte ASCII-Hexadecimal pair. Any applicable Message ID fields are two-byte ASCII-Hexadecimal values (two hex pairs).
  
\section{The Checksum}
  The checksum used is a NMEA checksum of all characters beginning with the \texttt{!} up to the last \texttt{,} preceeding the checksum.
  The start character \textbf{IS} included in the checksum. The last separator (\texttt{,}) \textbf{IS} also included in the checksum. In
  this sense, the checksum includes every character preceeding it in the entire message.
  The process of calculating the checksum is to merely \texttt{XOR} each byte with the previous byte.
  The calculated checksum is one byte encoded in ASCII-Hexadecimal. It immediately follows the last separator and preceeds the stop character.
  
  
  \large NOTE: \normalsize In this document, the example checksums used will be a random hex pair and are not actually valid.
\section{QUERY Messages}
  The purpose of a \mquery message is to allow the base station to request specific information from the satellite.
  Each query message has a corresponding \mresult message. The subtype name of the corresponding \mresult
  is the same as the subtype name of the \mquery.
  
  All \mquery messages require a Type, Subtype, and Checksum field. Some messages require additional fields following
  the subtype and preceeding the checksum.
  \begin{center}
    \begin{tabular}{| l | l | l | c |}
      \hline
      Subtype & Definition & Additional Fields & Total Field Count \\ \hline
      \texttt{HELLO} & Hello World & -- & 3 \\
      \texttt{POW\_PANEL} & Panel information query & Axis & 4 \\
      \texttt{POW\_BUS} & Power bus information query & -- & 3 \\
      \texttt{POW\_BATTERY} & Battery information query &  Battery & 4 \\
      \hline
    \end{tabular}
  \end{center}
  
  \subsection{QUERY HELLO}
    A \mquery \texttt{HELLO} Message is a \mquery request for a Hello World response from the satellite. This is the most basic
    message exchange with the satellite. An example exchange is below: \\[5pt]
    The base station will send a request \mquery message: \\[5pt]
    \texttt{!QUERY,HELLO,A0\$} \\[5pt]
    The satellite will now send a reply \mresult message: \\[5pt]
    \texttt{!RESULT,HELLO,Hello World,C2\$} \\[5pt]
    
    
  \subsection{QUERY POW\_PANEL}
    A \mquery \texttt{POW\_PANEL} message requests power information for one of the three axis panels. There is one additional
    field required for a \mquery \texttt{POW\_PANEL} query: the axis field. The axis field is one of three valid axes:
    \begin{itemize}
    \item \texttt{X}
    \item \texttt{Y}
    \item \texttt{Z}
    \end{itemize}
    
  \subsection{QUERY POW\_BUS}
    This query requests bus information from the EPS. It has no additional fields.
    
  \subsection{QUERY POW\_BATTERY}
    This query requests information on one of two batteries. It has one additional field to determine which battery should be reported.
    Valid values for the Battery field are: 
    \begin{itemize}
    \item \texttt{0}
    \item \texttt{1}
    \end{itemize}
  
  \subsection{QUERY Exchange Details}
  A \mquery exchange is a two-step exchange. First, the base station sends a \mquery message and waits.
  Upon receiving and validating a \mquery message, the satellite will compose and transmit a \mresult message.
  
  The satellite \textbf{IS NOT} permitted to transmit multiple \mresult
   messages per \mquery received.
  
  The base station \textbf{IS} permitted to retry \mquery messages until a \mresult is received.
  
 

\section{RESULT Messages}
  \mresult messages are sent by the satellite in response to a query of the same subtype name. Each result message contains at least four fields.
  The Type, Subtype, and Checksum fields are required, and all \mresult messages have at least one additional field containing the actual result
  data. Certain result subtypes may return more than one data field. The meanings of these additional
  fields will be described in each \texttt{RESULT} section. All data fields directly preceed the checksum and are in the listed order.
  
  \begin{center}
    \begin{tabular}{| l | l | l | c |}
      \hline
      Subtype & Definition & Data Fields & Total field count \\ \hline
      \texttt{HELLO} & Hello World & ``Hello World'' & 4\\
      \texttt{POW\_PANEL} & Panel status & Axis, Voltage, -Current, +Current & 7\\
      \texttt{POW\_BUS} & Bus status & Battery Current, 5V Current, 3.3V Current & 6 \\
      \texttt{POW\_BATTERY} & Battery status & Battery, Temperature, Voltage, Direction, Current & 8 \\
      \hline
    \end{tabular}
  \end{center}
  
  \subsection{RESULT HELLO}
    The \texttt{HELLO} subtype is a response to a ``Hello World'' request from the base station.
    The single data field of a \texttt{HELLO} result always contains the string ``Hello World''.

    
  \subsection{RESULT POW\_PANEL}
    This result contains four data fields. The first is the axis requested in the original query.
    The voltage and current fields contain the corresponding power values, each as a two-byte integer encoded as two hex 
    pairs.
    
  \subsection{RESULT POW\_BUS}
    This result contains three data fields. Each current field contains the corresponding EPS value. Each is encoded as
    two hex-pairs representing one 16 bit integer.
    
  \subsection{RESULT POW\_BATTERY}
    This result contains five data fields. The first field is the battery, either \texttt{0} or \texttt{1}, depending
    on which was requested. The current, temperature, and voltage fields are two-byte integers represented as two hex-pairs.
    The direction field is either \texttt{D} or \texttt{C}, for discharge or charge.
    
\section{COMMAND Messages}
  \mcommand messages are sent by the ground station to request the satellite perform an action
  or a configuration change. Each \mcommand should be followed by a \macommand sent from the satellite
  to acknowledge the command was performed or a \merror telling why the satellite was unable to perform
  the command.
  
  \begin{center}
  \begin{tabular}{| l | l | l | c |}
    \hline
    Subtype & Description & Additional Fields & Field Count \\ \hline
    \texttt{BURN} & Deploy Antennas & -- & 3\\
    \texttt{POW\_PRINT} & Print EPS Data & -- & 3\\
    \hline
  \end{tabular}
  \end{center}
  
  \subsection{COMMAND BURN}
  This command manually deploys the antennas. The satellite should initiate a burn routine and immediately respond
  with a \macommand acknowledging the burn
  
  \subsection{COMMAND POW\_PRINT}
  This is a debug command. It causes the EPS driver to print all the housekeeping datat in ASCII directly 
  to the console.
  
  \subsection{COMMAND Exchange Details}
  A \mcommand exchange is a two-step exchange. First, the base station issues a \mcommand message and waits to receive either
  a \macommand or \merror \texttt{COMMAND}. The satellite, upon receiving a \mcommand message, should either transmit a \merror 
  \texttt{COMMAND}, or perform the command and return a \macommand.
  
  The satellite \textbf{IS NOT} permitted to transmit multiple \macommand messages per \mcommand received.
  
  The base station \textbf{MAY} retry \mcommand messages, but it is recommended to \mquery the satellite
  if possible to determive whether the command was not received, or the \macommand was simply missed.

\section{ACK\_COMMAND Messages}
  \macommand messages acknowledge that a command was received and will or has been performed. If
  the command cannot be performed, a \merror should be returned instead. Similar to \mresult messages,
  the subtype of a \macommand is identical to the subtype of the \mcommand sent.
  
  
  \begin{center}
  \begin{tabular}{| l |}
    \hline
    Subtype\\ \hline
    \texttt{BURN}\\
    \texttt{POW\_PRINT} \\
    \hline
  \end{tabular}
  \end{center}
  
  
  
\section{NACK\_ERROR Messages}
  \merror messages are sent by the satellite to inform the base station that a message was received, but an error
  occurred during parsing. Some \merror messages have four fields. This one additional field is a human-readable
  description following the error subtype.
  
  \begin{center}
    \begin{tabular}{| l | l | l |}
      \hline
      Subtype & Error & Description field \\ \hline
      \texttt{TYPE} & The message type could not be determined & No \\
      \texttt{SUBTYPE} & The message subtype could not be determined & No \\
      \texttt{LENGTH} & Invalid field count for the message type or subtype & No\\
      \texttt{CHECKSUM} & Invalid or missing checksum & No \\
      \texttt{PARAM} & Invalid field in \mquery or \mcommand & Yes \\
      \texttt{COMMAND} & Unable to perform a command & Yes \\
      \texttt{UNSPECIFIED} & Other error occurred & Yes \\
      \hline
    \end{tabular}
  \end{center}

\section{Version History}

\noindent{\large0.2 \normalsize} 

Created descriptions of COMMAND messages

Began table of COMMANDs

Began table of ACK\_COMMANDs

COMMAND BURN added

Removed TIME queries; not possible with this RTC

Added NACK\_ERROR reference

Added checksum description

Added full definitions for power related queries

\noindent{\large0.1 \normalsize}

Initial release.

    
\end{document}