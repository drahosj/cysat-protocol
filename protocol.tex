%commands.tex - command protocol document for CySat

% Jake Drahos - Januar 2014

\documentclass{article}
\usepackage{times}
\usepackage[margin=0.75in]{geometry}

\begin{document}

\title{CySat Command and Data Protocol Description}
\author{Jake Drahos \\ \texttt{drahos@iastate.edu}}
\date{\today\\v0.2}


\newcommand{\mquery}{\texttt{QUERY }}
\newcommand{\mresult}{\texttt{RESULT }}
\newcommand{\mcommand}{\texttt{COMMAND }}
\newcommand{\mdownlink}{\texttt{DOWNLINK }}
\newcommand{\macommand}{\texttt{ACK\_COMMAND }}
\newcommand{\madownlink}{\texttt{ACK\_DOWNLINK }}
\newcommand{\merror}{\texttt{NACK\_ERROR }}


\maketitle

\section{Introduction}
This document is the canonical reference for the communications protocol used for CySat I and
future satellites for the CySat project.
This document is currently a working draft. As such, it is incomplete. Additions will be made, however, major
changes to already established details will not occur except in large version number upgrades of this document (for example,
version 0.7 to 1.1, but not from 0.4 to 0.5).
Version history can be found in the last section of this document.

\section{Protocol Overview}
The protocol is based on the idea of \emph{messages}.
Messages are entirely in ASCII.
Each message has several \emph{fields}, each of which is separated by a comma.
Every message begins with a start character, and ends with a stop character.

The communications protocol involves several types of message:
\begin{center}
  \begin{tabular}{| l | l | l |}
    \hline
    Message Type & Usage & Origin \\ \hline
    \mquery & Query the satellite & Base \\
    \mresult & Contains return information from a query & Satellite \\ \hline
    \mcommand & Issues a command to the satellite & Base \\
    \macommand & Acknowledges that a command was performed & Satellite \\ \hline
    \mdownlink & Downlink packet & Satellite \\
    \madownlink & Acknowledges downlink packet & Base \\ \hline
    \merror & Acknowledges an invalid packet & Satellite \\
    \hline
  \end{tabular}
\end{center}
  
  
\section{Message String Format}
  A valid command string takes the following format: \\[5pt]
  \verb=!MESSAGE_TYPE,FIELD_1,FIELD_2,CHECKSUM$=\\[5pt]
  There are three reserved characters, which cannot appear anywhere in an ASCII command:
  \begin{itemize}
  \item \verb=!= --- The start character
  \item \verb=$= --- The stop character
  \item \verb=,= --- Field separator
  \end{itemize}
  
  
  Each different message type has different requirements for valid lengths.
  
  
\begin{center}
  \begin{tabular}{| l | l | c |}
    \hline
    Message Type & Minimum Fields & Count \\ \hline
    \mquery & Type, Query Subtype, Checksum & 3 \\
    \mcommand & Type, Command, ID, Checksum & 4 \\
    \mdownlink & Type, ID, Data, Checksum & 4 \\
    \mresult & Type, Query Subtype, Result, Checksum  & 4 \\
    \macommand & Type, Command, ID, Checksum & 4 \\
    \madownlink & Type, ID, Checksum & 3 \\
    \merror & Type, Subtype, Checksum & 3 \\
    \hline
  \end{tabular}
\end{center}

Note that these are minimum required fields. Some messages, such as \mquery and \mcommand may have
different required parameters depending on the exact command or query being executed.


  The checksum is a one-byte ASCII-Hexadecimal pair. Any applicable Message ID fields are two-byte ASCII-Hexadecimal values (two hex pairs).
  
\section{The Checksum}
  The checksum used is a NMEA checksum of all characters beginning with the \texttt{!} up to the last \texttt{,} preceeding the checksum.
  The start character \textbf{IS} included in the checksum. The last separator (\texttt{,}) \textbf{IS} also included in the checksum. In
  this sense, the checksum includes every character preceeding it in the entire message.
  The process of calculating the checksum is to merely \texttt{XOR} each byte with the previous byte.
  The calculated checksum is one byte encoded in ASCII-Hexadecimal. It immediately follows the last separator and preceeds the stop character.
  
  
  \large NOTE: \normalsize In this document, the example checksums used will be a random hex pair and are not actually valid.
\section{QUERY Messages}
  The purpose of a \mquery message is to allow the base station to request specific information from the satellite.
  Each query message has a corresponding \mresult message. The subtype name of the corresponding \mresult
  is the same as the subtype name of the \mquery.
  
  \begin{center}
    \begin{tabular}{| l | l | l | c |}
      \hline
      Subtype & Definition & Fields & Field Count \\ \hline
      \texttt{HELLO} & Hello World & Type, Subtype, Checksum & 3 \\
      \texttt{POW\_PANEL} & Panel information query & Type, Subtype, Axis, Checksum & 4 \\
      \texttt{POW\_BUS} & Power bus information query & Type, Subtype, Checksum & 3 \\
      \texttt{POW\_BATTERY} & Battery information query & Type, Subtype, Battery, Checksum & 4 \\
      \hline
    \end{tabular}
  \end{center}
  
  \subsection{QUERY HELLO}
    A \mquery \texttt{HELLO} Message is a \mquery request for a Hello World response from the satellite. This is the most basic
    message exchange with the satellite. An example exchange is below: \\[5pt]
    The base station will send a request \mquery message: \\[5pt]
    \texttt{!QUERY,HELLO,A0\$} \\[5pt]
    The satellite will now send a reply \mresult message: \\[5pt]
    \texttt{!RESULT,HELLO,Hello World,C2\$} \\[5pt]
    
    
  \subsection{QUERY POW\_PANEL}
    A \mquery \texttt{POW\_PANEL} message requests power information for one of the three axis panels. There is one additional
    field required for a \mquery \texttt{POW\_PANEL} query: the axis field. The axis field is one of three valid axes:
    \begin{itemize}
    \item \texttt{X}
    \item \texttt{Y}
    \item \texttt{Z}
    \end{itemize}
    
  \subsection{QUERY POW\_BUS}
    This query requests bus information from the EPS. It has no additional fields.
    
  \subsection{QUERY POW\_BATTERY}
    This query requests information on one of two batteries. It has one additional field to determine which battery should be reported.
    Valid values for the Battery field are: 
    \begin{itemize}
    \item \texttt{0}
    \item \texttt{1}
    \end{itemize}
  
  \subsection{QUERY Exchange Details}
  A \mquery exchange is a two-step exchange. First, the base station sends a \mquery message and waits.
  Upon receiving and validating a \mquery message, the satellite will compose and transmit a \mresult message.
  
  The satellite \textbf{IS NOT} permitted to transmit multiple \mresult
   messages per \mquery received.
  
  The base station \textbf{IS} permitted to retry \mquery messages until a \mresult is received.
  
 

\section{RESULT Messages}
  \mresult messages are sent by the satellite in response to a query of the same subtype name. Each result message contains at least four fields.
  At least one field contains the result data. Certain result subtypes may return more than one data field. The meanings of these additional
  fields will be described in each \texttt{RESULT} section.
  
  \begin{center}
    \begin{tabular}{| l | l | l | c |}
      \hline
      Subtype & Definition & Fields & Count \\ \hline
      \texttt{HELLO} & Hello World & Type, Subtype, Result Data, Checksum & 4\\
      \texttt{POW\_PANEL} & Panel status & Type, Subtype, Axis, Voltage, -Current, +Current, Checksum & 7\\
      \texttt{POW\_BUS} & Bus status & Type, Subtype, Battery Current, 5V Current, 3.3V Current, Checksum & 6 \\
      \texttt{POW\_BATTERY} & Battery status & Type, Subtype, Battery, Temperature, Voltage, Direction, Current, Checksum & 8 \\
      \hline
    \end{tabular}
  \end{center}
  
  \subsection{RESULT HELLO}
    The \texttt{HELLO} subtype is a response to a ``Hello World'' request from the base station.
    The single data field of a \texttt{HELLO} result always contains the string ``Hello World''.

    
  \subsection{RESULT POW\_PANEL}
    This result contains four data fields. The first is the axis requested in the original query.
    The voltage and current fields contain the corresponding power values, each as a two-byte integer encoded as two hex 
    pairs.
    
  \subsection{RESULT POW\_BUS}
    This result contains three data fields. Each current field contains the corresponding EPS value. Each is encoded as
    two hex-pairs representing one 16 bit integer.
    
  \subsection{RESULT POW\_BATTERY}
    This result contains five data fields. The first field is the battery, either \texttt{0} or \texttt{1}, depending
    on which was requested. The current, temperature, and voltage fields are two-byte integers represented as two hex-pairs.
    The direction field is either \texttt{D} or \texttt{C}, for discharge or charge.
    
\section{NACK\_ERROR Messages}
  \merror messages are sent by the satellite to inform the base station that a message was received, but an error
  occurred during parsing. All \merror messages are three fields in length, with only the Type, Subtype, and Checksum fields.
  
  \begin{center}
    \begin{tabular}{| l | l |}
      \hline
      Subtype & Error \\ \hline
      \texttt{NOSTART} & The message does not begin with a start character \\
      \texttt{TYPE} & The message type could not be determined \\
      \texttt{LENGTH} & Invalid field count for the message type or subtype \\
      \texttt{CHECKSUM} & Invalid or missing checksum \\
      \texttt{NUMBER} & Invalid number formatting or not a number \\
      \texttt{UNSPECIFIED} & Other error occurred \\
      \hline
    \end{tabular}
  \end{center}

\section{Version History}

\noindent{\large0.2 \normalsize} 

Removed TIME queries; not possible with this RTC
Added NACK\_ERROR reference
Added checksum description
Added full definitions for power related queries

\noindent{\large0.1 \normalsize}

Initial release.

    
\end{document}