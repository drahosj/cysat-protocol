%commands.tex - command protocol document for CySat

% Jake Drahos - Januar 2014

\documentclass{article}
%\usepackage{times}
\usepackage[margin=0.75in]{geometry}

\begin{document}

\title{CySat Command and Data Protocol Description}
\author{Jake Drahos \\ \texttt{drahos@iastate.edu}}
\date{\today\\v0.3}


\newcommand{\mquery}{\texttt{QUERY }}
\newcommand{\mresult}{\texttt{RESULT }}
\newcommand{\mcommand}{\texttt{COMMAND }}
\newcommand{\mdownlink}{\texttt{DOWNLINK }}
\newcommand{\macommand}{\texttt{ACK\_COMMAND }}
\newcommand{\madownlink}{\texttt{ACK\_DOWNLINK }}
\newcommand{\merror}{\texttt{NACK\_ERROR }}

\newcommand{\example}[1] {
\smallskip
	\texttt{Example: !#1\$}
	
}

\newcommand{\fields}[1] {
\medskip
	\texttt{Fields: #1}
	 
}


\maketitle

\section{Introduction}
This document is the canonical reference for the communications protocol used for CySat I and
future satellites for the CySat project.
This document is currently a working draft. As such, it is incomplete. Additions will be made, however, major
changes to already established details will not occur except in large version number upgrades of this document (for example,
version 0.7 to 1.1, but not from 0.4 to 0.5).
Version history can be found in the last section of this document.

\section{Protocol Overview}
The protocol is based on the idea of \emph{messages}.
Messages are entirely in ASCII, with the exception of \mdownlink messages, which
contain a binary field.
Each message has several \emph{fields}, each of which is separated by a comma.
Every message begins with a start character, and ends with a stop character.

The communications protocol involves several types of message:
\begin{center}
  \begin{tabular}{| l | l | l |}
    \hline
    Message Type & Usage & Origin \\ \hline
    \mquery & Query the satellite & Base \\
    \mresult & Contains return information from a query & Satellite \\ \hline
    \mcommand & Issues a command to the satellite & Base \\
    \macommand & Acknowledges that a command was performed & Satellite \\ \hline
    \mdownlink & Downlink packet & Satellite \\
    \madownlink & Acknowledges downlink packet & Base \\ \hline
    \merror & Acknowledges an invalid packet & Satellite \\
    \hline
  \end{tabular}
\end{center}
  
  
\section{Message String Format}
  A valid command string takes the following format: \\[5pt]
  \verb=!MESSAGE_TYPE,FIELD_1,FIELD_2,CHECKSUM$=\\[5pt]
  There are three reserved characters, which cannot appear anywhere in an ASCII command:
  \begin{itemize}
  \item \verb=!= --- The start character
  \item \verb=$= --- The stop character
  \item \verb=,= --- Field separator
  \end{itemize}
  
  
  Each different message type has different requirements for valid lengths.
  
  
\begin{center}
  \begin{tabular}{| l | l | c |}
    \hline
    Message Type & Minimum Fields & Count \\ \hline
    \mquery & Type, Query Subtype, Checksum & 3 \\
    \mcommand & Type, Command Subtype, Checksum & 3 \\
    \mdownlink & Type, ID, Size, Data, Checksum & 5 \\
    \mresult & Type, Query Subtype, Result, Checksum  & 4 \\
    \macommand & Type, Command Subtype, Checksum & 3 \\
    \madownlink & Type, ID, Checksum & 3 \\
    \merror & Type, Subtype, Checksum & 3 \\
    \hline
  \end{tabular}
\end{center}

Note that these are minimum required fields. Some messages, such as \mquery and \mcommand may have
different required parameters depending on the exact command or query being executed.

The first field of any message is known as the ``Type'' field, and corresponds to the message type.
The second field is often the message ``subtype''. For \mquery and \mcommand messages, this is the `command' or `query'
being performed. For \mresult and \macommand messages, this corresponds to the subtype of the message
being acknowledged or the query being answered. The last field is always a checksum.


  The checksum is a one-byte ASCII-Hexadecimal pair. Any applicable Message ID fields are two-byte ASCII-Hexadecimal values (two hex pairs).
  
\section{The Checksum}
  The checksum used is a NMEA checksum of all characters beginning with the \texttt{!} up to the last \texttt{,} preceeding the checksum.
  The start character \textbf{IS} included in the checksum. The last separator (\texttt{,}) \textbf{IS} also included in the checksum. In
  this sense, the checksum includes every character preceeding it in the entire message.
  The process of calculating the checksum is to merely \texttt{XOR} each byte with the previous byte.
  The calculated checksum is one byte encoded in ASCII-Hexadecimal. It immediately follows the last separator and preceeds the stop character.
  
  
  \large NOTE: \normalsize In this document, the example checksums used will be a random hex pair and are not actually valid.
  
\pagebreak
\section{QUERY Messages}
  The purpose of a \mquery message is to allow the base station to request specific information from the satellite.
  Each query message has a corresponding \mresult message. The subtype name of the corresponding \mresult
  is the same as the subtype name of the \mquery.
  
  All \mquery messages require a Type, Subtype, and Checksum field. Some messages require additional fields following
  the subtype and preceeding the checksum.
  \begin{center}
    \begin{tabular}{| l | l | l | c |}
      \hline
      Subtype & Definition & Additional Fields & Total Field Count \\ \hline
      \texttt{HELLO} & Hello World & -- & 3 \\
      \texttt{POW\_PANEL} & Panel information query & Axis & 4 \\
      \texttt{POW\_BUS} & Power bus information query & -- & 3 \\
      \texttt{POW\_BATTERY} & Battery information query &  Battery & 4 \\
      \texttt{FOOTPRINTS} & Number of stored footprints & -- & 3\\
      \texttt{TIME} & Current mission time & -- & 3\\
      \hline
    \end{tabular}
  \end{center}
  
  \subsection{QUERY HELLO}
    A \mquery \texttt{HELLO} Message is a \mquery request for a Hello World response from the satellite. This is the most basic
    message exchange with the satellite. An example exchange is below: \\[5pt]
    The base station will send a request \mquery message: \\[5pt]
    \texttt{!QUERY,HELLO,A0\$} \\[5pt]
    The satellite will now send a reply \mresult message: \\[5pt]
    \texttt{!RESULT,HELLO,Hello World,C2\$} \\[5pt]
    
    
  \subsection{QUERY POW\_PANEL}
    A \mquery \texttt{POW\_PANEL} message requests power information for one of the three axis panels. There is one additional
    field required for a \mquery \texttt{POW\_PANEL} query: the axis field. The axis field is one of three valid axes:
    \begin{itemize}
    \item \texttt{X}
    \item \texttt{Y}
    \item \texttt{Z}
    \end{itemize}
    
    \fields{Type, Subtype, Axis, Checksum}   
    \example{QUERY,POW\_PANEL,X,1C}
    
  \subsection{QUERY POW\_BUS}
    This query requests bus information from the EPS. It has no additional fields.
    
    \fields{Type, Subtype, Checksum}
    \example{QUERY,POW\_BUS,2D}
    
  \subsection{QUERY POW\_BATTERY}
    This query requests information on one of two batteries. It has one additional field to determine which battery should be reported.
    Valid values for the Battery field are: 
    \begin{itemize}
    \item \texttt{0}
    \item \texttt{1}
    \end{itemize}
    
    \fields{Type, Subtype, Battery, Checksum}
    \example{QUERY,POW\_BATTERY,0,3D}
    
  \subsection{QUERY FOOTPRINTS}
    Requests the number of stored footprints for downlink.
    
    \fields{Type, Subtype, Checksum}
    \example{QUERY,FOOTPRINTS,4E}
    
  \subsection{QUERY TIME}
    Requests the mission time.
    
    \fields{Type, Subtype, Checksum}
    \example{QUERY,TIME,5F}
  
  \subsection{QUERY Exchange Details}
  A \mquery exchange is a two-step exchange. First, the base station sends a \mquery message and waits.
  Upon receiving and validating a \mquery message, the satellite will compose and transmit a \mresult message.
  
  The satellite \textbf{IS NOT} permitted to transmit multiple \mresult
   messages per \mquery received.
  
  The base station \textbf{IS} permitted to retry \mquery messages until a \mresult is received.
  
 
\pagebreak
\section{RESULT Messages}
  \mresult messages are sent by the satellite in response to a query of the same subtype name. Each result message contains at least four fields.
  The Type, Subtype, and Checksum fields are required, and all \mresult messages have at least one additional field containing the actual result
  data. Certain result subtypes may return more than one data field. The meanings of these additional
  fields will be described in each \texttt{RESULT} section. All data fields directly preceed the checksum and are in the listed order.
  
  \begin{center}
    \begin{tabular}{| l | l | l | c |}
      \hline
      Subtype & Definition & Data Fields & Total field count \\ \hline
      \texttt{HELLO} & Hello World & ``Hello World'' & 4\\
      \texttt{POW\_PANEL} & Panel status & Axis, Voltage, -Current, +Current & 7\\
      \texttt{POW\_BUS} & Bus status & Battery Current, 5V Current, 3.3V Current & 6 \\
      \texttt{POW\_BATTERY} & Battery status & Battery, Temperature, Voltage, Direction, Current & 8 \\
      \texttt{FOOTPRINTS} & Number of stored footprints & Footprints & 4\\
      \texttt{TIME} & Mission time & Time & 4\\
      \hline
    \end{tabular}
  \end{center}
  
  \subsection{RESULT HELLO}
    The \texttt{HELLO} subtype is a response to a ``Hello World'' request from the base station.
    The single data field of a \texttt{HELLO} result always contains the string ``Hello World''.

	\fields{Type, Subtype, Hello World, Checksum}
   \example{RESULT,HELLO,Hello World,A6}
    
  \subsection{RESULT POW\_PANEL}
    This result contains four data fields. The first is the axis requested in the original query.
    The voltage and current fields contain the corresponding power values, each as a two-byte integer encoded as two hex 
    pairs.
    
   \fields{Type, Subtype, Axis, Voltage, -Current, +Current, Checksum}
   \example{RESULT,POW\_PANEL,X,12FE,43AB,11CC,B7}
    
  \subsection{RESULT POW\_BUS}
    This result contains three data fields. Each current field contains the corresponding EPS value. Each is encoded as
    two hex-pairs representing one 16 bit integer.
    
    \fields{Type, Subtype, Battery Current, 5V Current, 3.3V Current, Checksum}
    \example{RESULT,POW\_BUS,B3D4,AA12,BB34,C8}
    
  \subsection{RESULT POW\_BATTERY}
    This result contains five data fields. The first field is the battery, either \texttt{0} or \texttt{1}, depending
    on which was requested. The current, temperature, and voltage fields are two-byte integers represented as two hex-pairs.
    The direction field is either \texttt{D} or \texttt{C}, for discharge or charge.
    
    \fields{Type, Subtype, Battery, Temperature, Voltage, Direction, Current, Checksum}
    \example{RESULT,POW\_BATTERY,0,0013,33C4,D,11B4,E9}
    
  \subsection{RESULT FOOTPRINTS}
  Returns the number of stored footprints. This number should be a four-byte big-endian
  integer, returned as four consecutive hex pairs.
  
   \fields{Type, Subtype, Footprints, Checksum}
   \example{RESULT,FOOTPRINTS,004B3F4C,F0}
  
  \subsection{RESULT TIME}
  Returns the mission time. This number should be a four-byte big-endian integer, 
  formatted as four consecutive hex pairs.
   
   \fields{Type, Subtype, Time, Checksum}
   \example{RESULT,TIME,0000443B,3A}
    
\pagebreak
\section{COMMAND Messages}
  \mcommand messages are sent by the ground station to request the satellite perform an action
  or a configuration change. Each \mcommand should be followed by a \macommand sent from the satellite
  to acknowledge the command was performed or a \merror telling why the satellite was unable to perform
  the command.
  
  \begin{center}
  \begin{tabular}{| l | l | l | c |}
    \hline
    Subtype & Description & Additional Fields & Field Count \\ \hline
    \texttt{BURN} & Deploy Antennas & -- & 3\\
    \texttt{POW\_PRINT} & Print EPS Data & -- & 3\\
    \texttt{DOWNLINK} & Initiate a downlink session & -- & 3\\
    \texttt{RESET\_CLOCK} & Reset mission clock to zero & -- & 3\\
    \texttt{SET\_CLOCK} & Set clock to provided time & Time & 4\\
    \texttt{REBOOT} & Perform a reboot & -- & 3\\
    \texttt{REBOOT\_HARD} & Perform an immediate reboot & -- & 3\\
    \hline
  \end{tabular}
  \end{center}
  
  \subsection{COMMAND BURN}
  This command manually deploys the antennas. The satellite should initiate a burn routine and immediately respond
  with a \macommand acknowledging the burn
  
  \fields{Type, Subtype, Checksum}
  \example{COMMAND,BURN,4B}
  
  \subsection{COMMAND POW\_PRINT}
  This is a debug command. It causes the EPS driver to print all the housekeeping datat in ASCII directly 
  to the console.
  
  \fields{Type, Subtype, Checksum}
  \example{COMMAND,POW\_PRINT,5C}
  
  \subsection{COMMAND DOWNLINK}
  This command orders the satellite to initiate a downlink session. The satellite must
  respond with an \macommand before sending \mdownlink messages. For more information,
  see the section titled ``Downlink Protocol.'' 
  
  \fields{Type, Subtype, Checksum}
  \example{COMMAND,DOWNLINK,6D}
  
  \subsection{COMMAND RESET\_CLOCK}
  This command resets the mission clock to zero. This should not cause the satellite
  to go through any first-boot initialization; it should simply set the clock to zero.
  
  \fields{Type, Subtype, Checksum}
  \example{COMMAND,RESET\_CLOCK,7E}
  
  \subsection{COMMAND SET\_CLOCK}
  This command sets the mission clock to the provided value. The mission clock measures
  seconds since mission time zero, with mission time zero being the time of first boot.
  The parameter should be a 4-byte big-endian number, consisting of 4 continuous hex
  pairs.
  
  \fields{Type, Subtype, Time, Checksum}
  \example{COMMAND,SET\_CLOCK,0001B217,8F}
  
  \subsection{COMMAND REBOOT}
  The satellite must transmit a \macommand message prior to the reboot. The satellite
  is permitted to finish any other operations prior to the reboot.
  
  \fields{Type, Subtype, Checksum}
  \example{COMMAND,REBOOT,A2}
  
  \subsection{COMMAND REBOOT\_HARD}
  The satellite must perform a hardware reset as soon as this message is parsed, without
  regard for any other tasks that may be in progress.
  
  \fields{Type, Subtype, Checksum}
  \example{COMMAND,REBOOT\_HARD,B3}
  
  \subsection{COMMAND Exchange Details}
  A \mcommand exchange is a two-step exchange. First, the base station issues a \mcommand message and waits to receive either
  a \macommand or \merror \texttt{COMMAND}. The satellite, upon receiving a \mcommand message, should either transmit a \merror 
  \texttt{COMMAND}, or perform the command and return a \macommand.
  
  The satellite \textbf{IS NOT} permitted to transmit multiple \macommand messages per \mcommand received.
  
  The base station \textbf{MAY} retry \mcommand messages, but it is recommended to \mquery the satellite
  if possible to determine whether the command was not received, or the \macommand was simply missed.
  
\pagebreak
\section{DOWNLINK Messages}
  The \mdownlink message is a special case because it is a hybrid ascii-binary 
  format. Because of this, it is the only message with a SIZE field.
  
  \begin{center}
  \begin{tabular}{| l | l | l |}
   \hline
  	Field & Purpose & Format \\
  	Type & DOWNLINK & Text \\
  	ID & Associate \mdownlink with \madownlink & Two bytes hex \\
  	Size & Size of payload & Two bytes hex \\
  	Data & Data payload & Binary data \\
  	Checksum & Message Checksum & Normal checksum format \\
   \hline
  \end{tabular}
  \end{center}
  	
  As \mdownlink messages contain a binary segment, the stop character is not
  a reliable way to determine the end of the message. Care must be taken to 
  properly take into account the size of the data payload.
 
  The message ID must be unique among all recent messages. For more
  information, see the section titled ``Downlink Protocol''
  
  \fields{Type, ID, Length, Binary Data, Checksum}
  \example{DOWNLINK,013B,0010,aB!12cOb,e\$h9E.e,D4}
  
  \pagebreak[0]
\section{ACK\_DOWNLINK Messages}
	The \madownlink message acknowledges a \mdownlink message and permits the
	satellite to discard the contained footprints or other downlink data. The 
	ID contained in the \madownlink message corresponds to the received and
	stored \mdownlink.
	
	\fields{Type, ID, Checksum}
	\example{ACK\_DOWNLINK,013B,E5}
	
\pagebreak[0]	
\section{ACK\_COMMAND Messages}
  \macommand messages acknowledge that a command was received and will or has been performed. If
  the command cannot be performed, a \merror should be returned instead. Similar to \mresult messages,
  the subtype of a \macommand is identical to the subtype of the \mcommand sent.
  
  
  \begin{center}
  \begin{tabular}{| l |}
    \hline
    Subtype\\ \hline
    \texttt{BURN}\\
    \texttt{POW\_PRINT} \\
    \texttt{DOWNLINK}\\
    \texttt{RESET\_CLOCK}\\
    \texttt{SET\_CLOCK}\\
    \texttt{REBOOT}\\
    \hline
  \end{tabular}
  \end{center}
  
  \fields{Type, Subtype, Checksum}
  \example{ACK\_COMMAND,BURN,F6}
  \example{ACK\_COMMAND,SET\_CLOCK,A7}
  
\pagebreak[4]
\section{NACK\_ERROR Messages}
  \merror messages are sent by the satellite to inform the base station that a message was received, but an error
  occurred during parsing. Some \merror messages have four fields. This one additional field is a human-readable
  description following the error subtype.
  
  \begin{center}
    \begin{tabular}{| l | l | l |}
      \hline
      Subtype & Error & Description field? \\ \hline
      \texttt{TYPE} & The message type could not be determined & No \\
      \texttt{SUBTYPE} & The message subtype could not be determined & No \\
      \texttt{LENGTH} & Invalid field count for the message type or subtype & No\\
      \texttt{CHECKSUM} & Invalid or missing checksum & No \\
      \texttt{PARAM} & Invalid field in \mquery or \mcommand & Yes \\
      \texttt{COMMAND} & Unable to perform a command & Yes \\
      \texttt{UNSPECIFIED} & Other error occurred & Yes \\
      \hline
    \end{tabular}
  \end{center}
  
  \fields{Type, Subtype, Checksum}
  \example{NACK\_ERROR,TYPE,B8}
  
  \fields{Type, Subtype, Description, Checksum}
  \example{NACK\_ERROR,PARAM,W is not an axis,D9}
 

\pagebreak[1]
\section{Downlink Protocol}
  During downlink, the satellite operates in a slightly different manner.
  Upon receiving a the to begin a downlink session, the satellite will begin to
  transmit \mdownlink packets. The downlink session ends when one of two
  conditions is met: The satellite runs out of footprints, or the satellite times
  out on waiting for a downlink packet to be acknowledged. The details of these
  are covered in the subsection titled ``Cleanup''
  
  The rate of transmission is determined by the satellite. Messages should be
  transmitted fast enough to ensure continuous communication, but the rate of 
  arrival of \madownlink messages should be taken into account to ensure that
  messages are not likely to be dropped.
  
  The satellite should pull data from non-volatile memory and cache it for
  transmission. Should the transmission be successfully acknowledged, the
  data should be discarded. If the transmission times out, the data must be 
  returned to safe storage.  
  
  \subsection{Cleanup}
    If the session ends due to the satellite exhausting its supply of buffered
    data, it simply ceases transmission of \mdownlink messages. Upon receipt of 
    the last \madownlink, the satellite can safely leave downlink mode. The
    station can confirm that the satellite has no more data to transmit via a 
    \mquery. Should the satellite transmit all available downlink data, but
    fail to receive a needed \madownlink, the session ends exactly as if the
    time-out had occurred at any other point.
    
    Should the session end due to a time-out, the satellite immediately ceases
    transmission of new \mdownlink messages and begin waiting for a time-out of
    any currently transmitted but not yet acknowledged messages. Any \madownlink
    messages received during this time are taken into account and handled
    properly. At the end of the last time-out, any unacknowledged data is safely
    returned to non-volatile memory and the satellite leaves downlink mode until
    a new downlink session is initialized.

\newpage
\section{Version History}

\noindent{\large0.3 \normalsize}

Added fields and examples to each message subtype

Added definition of downlink messages and protocol

Added many commands, including time, reboot, and downlink.

Added several queries, including time and footprint queries

\noindent\line(1,0){250}

\noindent{\large0.2 \normalsize} 

Created descriptions of COMMAND messages

Began table of COMMANDs

Began table of ACK\_COMMANDs

COMMAND BURN added

Removed TIME queries; not possible with this RTC

Added NACK\_ERROR reference

Added checksum description

Added full definitions for power related queries

\noindent\line(1,0){250}

\noindent{\large0.1 \normalsize}

Initial release.

    
\end{document}